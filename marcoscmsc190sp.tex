\documentclass[journal]{./IEEE/IEEEtran}
\usepackage{cite,graphicx}

\newcommand{\SPTITLE}{Room-8! - A Matchmaking Application for Finding College Roommates}
\newcommand{\ADVISEE}{Joseph Matthew R. Marcos}
\newcommand{\ADVISER}{Caroline Natalie M. Peralta}

\newcommand{\BSCS}{Bachelor of Science in Computer Science }
\newcommand{\ICS}{Institute of Computer Science}
\newcommand{\UPLB}{University of the Philippines Los Ba\~{n}os }
\newcommand{\REMARK}{\thanks{Presented to the Faculty of the \ICS, \UPLB\
                            in partial fulfillment of the requirements
                            for the Degree of \BSCS }}

\markboth{CMSC 190 Special Problem, \ICS}{}
\title{\SPTITLE}
\author{\ADVISEE~and~\ADVISER%
\REMARK
}
\pubid{\copyright~2016~ICS \UPLB}
% \IEEEpubid{\begin{minipage}{\textwidth}\ \\[12pt] \centering
%     \copyright~2016~ICS \UPLB
% \end{minipage}}

%%%%%%%%%%%%%%%%%%%%%%%%%%%%%%%%%%%%%%%%%%%%%%%%%%%%%%%%%%%%%%%%%%%%%%%%%%

\begin{document}

% TITLE
\maketitle

% ABSTRACT
\begin{abstract}
% The abstract should be \textit{informational}. Typically a single paragraph
% of about fifty to two hundred workds, the abstract allows your readers to judge
% whether or not the article is of relevance to them. It should therefore be
% a concise summary of the aims, scope, and conclusions of your work. There
% is no space for unnecessary texts; an abstract should be kept to as few words
% as possible while remaining reasonably informative. Irrelevancies, such as
% minor details or a \textit{description} of the structure of the paper, are
% inappropriate, as are acronyms, abbreviations, and mathematics. Sentences such
% as ``we review relevant literature" should be omitted. ~\cite{01}.

This article explores the role and significance of roommates in a college student's development - from the initial adaptation with the roommate setting, to his/her academic performance and life after college. We highlight the importance of finding the proper roommates so that a student can maximize his/her psychological development, get higher grades, and develop open-mindedness. We also propose a solution in the form of a web application that can help \UPLB students find mutually beneficial roommates for their college life. This solution also extends itself to solve the dorm-finding problem students experience at the start of every semester.
\end{abstract}

% INDEX TERMS
% \begin{keywords}
% key, words, separated, by, comma
% \end{keywords}

% INTRODUCTION
\section{Introduction}
% To be effective, the introduction should answer the questions ``Why and What For (Four)?" Expanded, these questions are----
% \subsection{Why is the topic of interest?}

% Expound more on the psychological importance of college
Several studies support that social functioning during college plays a major role in a student's psychological development. According to Chickering and Reisser, one of the seven vectors of psychological developmental issues that college students face is cultivating mature interpersonal relationships\cite{chickering}. College-aged students also fall within Arnett's emerging adulthood stage, where identity formation is associated with friendships\cite{erb}. According to Erikson's stage theory of psychological development, a young adult prefers to seek intimacy in relationships rather than in isolation\cite{erikson}.
\\
\\
\indent Studies have linked social functioning to mental health and adjustment to college life \cite{erb}. A plethora of benefits such as psychological well-being, environmental mastery, personal growth, and self-acceptance have all been related to a student's ability to form meaningful relationships\cite{erb}. The conjunction between a person’s social capabilities and these benefits is so strong that we can use it to predict how well a student  will adjust to the new demands in college life. An individual who can develop good relationships in college has less problems.

    \subsection{Why College Roommate Relationships}
    Our goal is to make it easier for students to acquire these benefits with less guess work. We want students to develop better interpersonal relationships in college - particularly roommate relationships because these are relationships that are widely experienced by many college students\cite{erb}. Roommates have frequent contact, negotiations of responsibilities and compromises about the living environment, which often include noise level, cleanliness level, sleep/waking hours, visitors, decorations, and bills\cite{erb}. These may become challenges when dealing with each other.
    \\
    \\
    \indent For many college students, many ``firsts'' add to the challenges of campus life. Usually, roommates are the first ``people of equal status'' they live with, compared to the parent-child relationship they usually have experienced at home\cite{erb}.
    \\
    \\
    \indent Unlike friendships, college students may often not choose roommates, causing personality mismatches. Many studies support that roommates, while having the potential to be a boon to a student, can also be a hindrance to having a quality college life. According to Erb\cite{erb}, citing Liu\cite{liu}, in a 31,500 sample sized survey in the United States of America, 50.1\% of women and 44.1\% of men experienced “frequent” conflicts with roommates or housemates. In another survey in the same country, 5.6\% of undergraduates reported that difficulties with their roommates have hampered their academic endeavors\cite{erb}. For comparison, 4\% of American students said that alcohol did the same thing\cite{erb}.
    \\
    \\
    \indent We target college roommates because it has been proven that college roommate relationships have a momentous effect on a student's performance before and after graduation. Although the college-roommate phenomenon has existed for a long time, there is only one paper that congregates empirical information about college-roommate relationships\cite{erb}.

    \subsection{Proposed Solution}
    In order to maximize the number of compatible roommates on campus, we will be making a roommate-matching application for UPLB. Unlike other roommate matching applications that exist today, we will target students of \UPLB that are looking for roommates. Students of \UPLB  have different needs and preferences to working class people, so we cannot use existing roommate matching applications. In addition, there are no roommate matching applications that cater to the UPLB area or the Philippines. It is about time we have one for campus.
% \pubidadjcol <- to avoid overlapping of pubid to content
% I ended up putting the new section to a new page since only the title stayed in the first page anyway
% Similar Applications
\newpage
\section{Similar Applications}
\pubidadjcol
There are existing web applications that help users find other roommates. Two examples of these are \textit{ Roommiematch} and \textit{Roommates.com}. Both of them cater to long term tenants at first world countries such as Canada, Australia, and USA.
\\
\\
These applications have more similarities than differences. They ask the user if they are looking for a place to stay, or if they have a room and are looking for someone to stay with them. Then, the user is bombarded by a long questionnaire that he has to fill out. These questions cover most of the basic criteria you would have when looking for a room.
\\
\\
\indent Both applications have similar questions, but have subtle insignificant differences. Roommie Match does not require the user to log in before filling out the form, but in order to get a match, you will have to provide your credentials. Some people may find this surreptitious, as users will likely commit to giving their credentials since they have already filled out a long form. Roommates.com is less sneaky, as you are required to log in even before you start filling out the questionnaire.

\subsection{Roommate.com}
Roommate.com can be accessed by going to \textit{www.roommates.com}. When you are registering, it asks if you are looking for a room or if you have a room. If you are looking for a room it asks in which city and state you are looking for. Then it asks you how far from the city in miles you are willing to live in. Then it shows a list of cities which fit the criteria. You can select which cities it will search. Afterwards, you can select the regions of those cities it will look up (North, South, West, East, Central). Then it asks your preferred residence type (house, apartment, townhouse/condo, other), your move-in date, your maximum rent budget, duration, and whether or not you are willing to share a bedroom or a bathroom. After that, it asks about your preferred roommate characteristics - age, gender (all of it and all possible restrictions in the rainbow), smoking habits, cleanliness, pets (dogs, cats, caged animals, etc), children (if you are willing to live with children or not). Finally, it asks details about yourself -  age, sex, sexual orientation, occupation, smoking habits {indoor, outdoor}*, cleanliness, pets, children, and which emoticon in their list best represents you.
\\
\\
Once logged in as a user looking for a room, it shows a list of matching people that have rooms. You can see their age range, smoking habits, payment scheme (monthly, down payment, cost), location, availability date, utilities (electricity, gas, water), furnishing, nearby landmarks, nearby public transport hubs, existing occupants, etc. You can also see the profile of the person who owns the room. Premium members of the website can comment on profiles of other people.
\\
\\
There are pictures of the apartment, and the person looking for the apartment can post a picture of him/herself. A user can also opt to select a smiley face that best represents his personality.
\subsection{Roommiematch}
Roommiematch is another roommate match making algorithm that caters to some First-world countries.


% MATERIALS AND METHODS
\section{Materials and Methods}
% Irving's Algorithm
% Gale-Shapley's Algorithm

% RESULTS AND DISCUSSION
\section{Results and Discussion}

% CONCLUSION AND FUTURE WORK
\section{Conclusion and Future Work}

% APPENDICES
\appendices

\section{Proof of methods down here}

\newpage
\section{Tables, Lists, and Figures}
\newpage
% ACKNOWLEDGMENT
\section*{Acknowledgment}
I would like to thank my adviser \ADVISER for her patience.

% BIOGRAPHY
\begin{biography}[{\includegraphics{./yourPicture.eps}}]{\ADVISEE}
\end{biography}


\newpage
% BIBLIOGRAPHY
% \bibliography{./cs190-ieee}
\bibliographystyle{./IEEE/IEEEtran}
\bibliography{marcoscmsc190sp}
% \nocite{*}
\end{document}

