\documentclass[journal]{./IEEE/IEEEtran}
\usepackage{cite,graphicx}

\newcommand{\SPTITLE}{Room-8! - A Matchmaking Application for Finding College Roommates}
\newcommand{\ADVISEE}{Joseph Matthew R. Marcos}
\newcommand{\ADVISER}{Caroline Natalie M. Peralta}

\newcommand{\BSCS}{Bachelor of Science in Computer Science}
\newcommand{\ICS}{Institute of Computer Science}
\newcommand{\UPLB}{University of the Philippines Los Ba\~{n}os}
\newcommand{\REMARK}{\thanks{Presented to the Faculty of the \ICS, \UPLB\
                             in partial fulfillment of the requirements
                             for the Degree of \BSCS}}

\markboth{CMSC 190 Special Problem, \ICS}{}
\title{\SPTITLE}
\author{\ADVISEE~and~\ADVISER%
\REMARK
}
\pubid{\copyright~2006~ICS \UPLB}

%%%%%%%%%%%%%%%%%%%%%%%%%%%%%%%%%%%%%%%%%%%%%%%%%%%%%%%%%%%%%%%%%%%%%%%%%%

\begin{document}

% TITLE
\maketitle

% ABSTRACT
\begin{abstract}
% The abstract should be \textit{informational}. Typically a single paragraph
% of about fifty to two hundred workds, the abstract allows your readers to judge
% whether or not the article is of relevance to them. It should therefore be
% a concise summary of the aims, scope, and conclusions of your work. There
% is no space for unnecessary texts; an abstract should be kept to as few words
% as possible while remaining reasonably informative. Irrelevancies, such as
% minor details or a \textit{description} of the structure of the paper, are
% inappropriate, as are acronyms, abbreviations, and mathematics. Sentences such
% as ``we review relevant literature" should be omitted. ~\cite{01}.

This article explores the role and importance of roommates in a college student$'$s development - from the initial adaptation in the roommate setting, to their grades and life after college. We highlight the importance of finding the proper roommates and propose a solution in the form of a web application to help UPLB students find mutually beneficial roommates for their college life. This solution also extends itself to solve the \textit{dorm-finding} problem we experience at the start of every semester.
\end{abstract}

% INDEX TERMS
% \begin{keywords}
% key, words, separated, by, comma
% \end{keywords}

% INTRODUCTION
\section{Introduction}
% To be effective, the introduction should answer the questions ``Why and What For (Four)?" Expanded, these questions are----
    % \subsection{Why is the topic of interest?}

    % Expound more on the psychological importance of college
    Social functioning during college plays a major role in a student$'$s psychological development. According to Chickering and Reisser, one of the seven vectors of psychological developmental issues that college students face is cultivating mature interpersonal relationships\cite{chickering}. College-aged students also fall within Arnett$'$s emerging adulthood stage, where identity formation associated with friendships take place \cite{erb}. According to Erikson$'$s stage theory of psychological development, a young adult prefers to seek intimacy in relationships rather than isolation\cite{erikson}.
    \\
    \\
    Studies have linked social functioning to mental health and adjustment to college life \cite{erb}. A plethora of benefits such as psychological well-being, environental mastery, personal growth, and self-acceptance have all been linked to a student$'$s ability to form relationships\cite{erb}. A person


\subsection{Why College Roommate Relationships}
    We can help students obtain these benefits by improving college roommate relationships because it is a special type of interpersonal relationship that is widely and uniquely experienced by college students\cite{erb}.

    We target college roommates because it has been proven that college roommate relationships have a momentous effect on a student$’$s performance before and after graduation. We put more effort in helping college students find the roommates that will help improve themselves.
    \\
    \\
    Having the correct college roommate can affect a college student in both his/her personal and academic life. A research paper suggests that roommate contact with different people can potentially diminish prejudices given that there are a few conditions that are met. Waldo in 1986 found that there is a correlation between positive college-roommate interactions and higher GPAs. Inversely, Dusselier, Dunn, Wang, Shelley, and Whalen noted that frequent conflicts can be used to determine the overall stress level of an individual\cite{dusselier}. Choosing a college roommate that does not match with an individual can have detrimental effects. According to Keup, having a bad relationship with college roommates can cause one of the greatest disappointments of a student’s freshman year, and can decrease his/her overall satisfaction\cite{keup}.
    \\
    \\
    % \subsection{What is the background on the previous solutions, if any?}
    % \subsection{What is the background on potential solutions}
    % \subsection{What was attempted in the present effort (research project)?}
    % \subsection{What will be presented in this paper?}
    It is important that we have a tool to properly match college roommates. Bradbury and Mather discovered that 3 of 4 college students who lived on campus had difficulties with their college roommate. In fact, one of their participants mentioned that a bad relationship with her college roommate was a factor in her decision
    to leave the university\cite{bradbury}.

% MATERIALS AND METHODS
\section{Materials and Methods}

% RESULTS AND DISCUSSION
\section{Results and Discussion}
% CONCLUSION AND FUTURE WORK
\section{Conclusion and Future Work}
% APPENDICES
\appendices

\section{Proof of methods down here}

\section{}

% ACKNOWLEDGMENT
\section*{Acknowledgment}
Many thanks to... my mother, my father, and all the people that inspired me to do this very short paper.

% BIOGRAPHY
\begin{biography}[{\includegraphics{./yourPicture.eps}}]{\ADVISEE}
\end{biography}


\newpage
% BIBLIOGRAPHY
% \bibliography{./cs190-ieee}
\bibliographystyle{./IEEE/IEEEtran}
\bibliography{marcoscmsc190sp}
% \nocite{*}
\end{document}

