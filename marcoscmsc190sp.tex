\documentclass[journal]{./IEEE/IEEEtran}
\usepackage{cite,graphicx}

\newcommand{\SPTITLE}{Your SP Title Goes Here}
\newcommand{\ADVISEE}{Joseph Matthew R. Marcos}
\newcommand{\ADVISER}{Caroline Natalie M. Peralta}

\newcommand{\BSCS}{Bachelor of Science in Computer Science}
\newcommand{\ICS}{Institute of Computer Science}
\newcommand{\UPLB}{University of the Philippines Los Ba\~{n}os}
\newcommand{\REMARK}{\thanks{Presented to the Faculty of the \ICS, \UPLB\
                             in partial fulfillment of the requirements
                             for the Degree of \BSCS}}

\markboth{CMSC 190 Special Problem, \ICS}{}
\title{\SPTITLE}
\author{\ADVISEE~and~\ADVISER%
\REMARK
}
\pubid{\copyright~2006~ICS \UPLB}

%%%%%%%%%%%%%%%%%%%%%%%%%%%%%%%%%%%%%%%%%%%%%%%%%%%%%%%%%%%%%%%%%%%%%%%%%%

\begin{document}

% TITLE
\maketitle

% ABSTRACT
\begin{abstract}
% The abstract should be \textit{informational}. Typically a single paragraph
% of about fifty to two hundred workds, the abstract allows your readers to judge
% whether or not the article is of relevance to them. It should therefore be
% a concise summary of the aims, scope, and conclusions of your work. There
% is no space for unnecessary texts; an abstract should be kept to as few words
% as possible while remaining reasonably informative. Irrelevancies, such as
% minor details or a \textit{description} of the structure of the paper, are
% inappropriate, as are acronyms, abbreviations, and mathematics. Sentences such
% as ``we review relevant literature" should be omitted. ~\cite{01}.

This article explores the role and importance of roommates in college students' \textit{in UPLB} development, from the initial adaptation in the roommate setting, to their grades, and to their life after college. We highlight the importance of finding the proper roommates and propose a solution in the form of a web application to help UPLB students find the proper roommates for their college life. This solution also extends itself to solve the \textit{dorm-finding} problem we experience. ~\cite{01}
\end{abstract}

% INDEX TERMS
% \begin{keywords}
% key, words, separated, by, comma
% \end{keywords}

% INTRODUCTION
\section{Introduction}
To be effective, the introduction should answer the questions ``Why and What For (Four)?" Expanded, these questions are----

\subsection{Why is the topic of interest?}
Start paragraph here...

\subsection{What is the background on the previous solutions, if any?}
Start of first paragraph. The quick brown fox jumps over the lazy dog. The quick
brown fox jumps over the lazy dog. The quick brown fox jumps over the lazy dog. The
quick brown fox jumps over the lazy dog.

Start of second paragraph. The quick brown fox jumps over the lazy dog. The quick
brown fox jumps over the lazy dog. The quick brown fox jumps over the lazy dog. The
quick brown fox jumps over the lazy dog.

\subsection{What is the background on potential solutions}

\subsection{What was attempted in the present effor (research project)?}

\subsection{What will be presented in this paper?}
Subsection text here.

% MATERIALS AND METHODS
\section{Materials and Methods}

\subsubsection{First Heading}
\subsubsection{Second Heading}

% RESULTS AND DISCUSSION
\section{Results and Discussion}
% CONCLUSION AND FUTURE WORK
\section{Conclusion and Future Work}
% APPENDICES
\appendices

\section{Proof of methods down here}

\section{}

% ACKNOWLEDGMENT
\section*{Acknowledgment}
Many thanks to... my mother, my father, and all the people that inspired me to do this very short paper.

% BIOGRAPHY
\begin{biography}[{\includegraphics{./yourPicture.eps}}]{\ADVISEE}
\end{biography}


\newpage
% BIBLIOGRAPHY
% \bibliography{./cs190-ieee}
\bibliographystyle{./IEEE/IEEEtran}
\bibliography{marcoscmsc190sp}
% \nocite{*}
\end{document}

