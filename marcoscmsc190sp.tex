% Remove 12pt, change draftcls -> final when doing final.
\documentclass[journal]{./IEEE/IEEEtran}
\usepackage{cite,graphicx}

\newcommand{\SPTITLE}{Room-8! - A Matchmaking Application for Finding College Roommates}
\newcommand{\ADVISEE}{Joseph Matthew R. Marcos}
\newcommand{\ADVISER}{Caroline Natalie M. Peralta}

\newcommand{\BSCS}{Bachelor of Science in Computer Science }
\newcommand{\ICS}{Institute of Computer Science}
\newcommand{\UPLB}{University of the Philippines Los Ba\~{n}os }
\newcommand{\REMARK}{\thanks{Presented to the Faculty of the \ICS, \UPLB\
                            in partial fulfillment of the requirements
                            for the Degree of \BSCS }}

\markboth{CMSC 190 Special Problem, \ICS}{}
\title{\SPTITLE}
\author{\ADVISEE~and~\ADVISER%
\REMARK
}
\pubid{\copyright~2016~ICS \UPLB}

%%%%%%%%%%%%%%%%%%%%%%%%%%%%%%%%%%%%%%%%%%%%%%%%%%%%%%%%%%%%%%%%%%%%%%%%%%

\begin{document}

% TITLE
\maketitle

% ABSTRACT
\begin{abstract}
This article explores the role and significance of roommates in a college student's development - from the initial adaptation with the roommate setting, to his/her academic performance and life after college. We highlight the importance of finding the proper roommates so that a student can maximize his/her psychological development, get higher grades, and develop open-mindedness. We also propose a solution in the form of a web application that can help \UPLB students find mutually beneficial roommates for their college life. This solution also extends itself to solve the dorm-finding problem students experience at the start of every semester.
\end{abstract}

% INDEX TERMS
\begin{keywords}
UPLB, University of the Philippines Los Ba\~{n}os, Roommate, Roommate matching
\end{keywords}

% INTRODUCTION
\section{Introduction}
% To be effective, the introduction should answer the questions ``Why and What For (Four)?" Expanded, these questions are----
% \subsection{Why is the topic of interest?}

% Expound more on the psychological importance of college
Several studies support that social functioning during college plays a major role in a student's psychological development. According to Chickering and Reisser, one of the seven vectors of psychological developmental issues that college students face is cultivating mature interpersonal relationships\cite{chickering}. College-aged students also fall within Arnett's emerging adulthood stage, where identity formation is associated with friendships\cite{erb}. According to Erikson's stage theory of psychological development, a young adult prefers to seek intimacy in relationships rather than in isolation\cite{erikson}.

Studies have linked social functioning to mental health and adjustment to college life \cite{erb}. A plethora of benefits such as psychological well-being, environmental mastery, personal growth, and self-acceptance have all been related to a student's ability to form meaningful relationships\cite{erb}. The conjunction between a person’s social capabilities and these benefits is so strong that we can use it to predict how well a student  will adjust to the new demands in college life. An individual who can develop good relationships in college has less problems.
    \subsection{Why College Roommate Relationships}
    Our goal is to make it easier for students to acquire these benefits with less guess work. We want students to develop better interpersonal relationships in college $-$ particularly roommate relationships because these are relationships that are widely experienced by many college students\cite{erb}. Roommates have frequent contact, negotiations of responsibilities and compromises about the living environment, which often include noise level, cleanliness level, sleep/waking hours, visitors, decorations, and bills\cite{erb}. These may become challenges when dealing with each other.

    For many college students, many \"firsts\" add to the challenges of campus life. Usually, roommates are the first \"people of equal status\" they live with, compared to the parent-child relationship they usually have experienced at home\cite{erb}.

    Unlike friendships, college students may often not choose roommates, causing personality mismatches. Many studies support that roommates, while having the potential to be a boon to a student, can also be a hindrance to having a quality college life. According to Erb\cite{erb}, citing Liu\cite{liu}, in a 31,500 sample sized survey in the United States of America, 50.1\% of women and 44.1\% of men experienced “frequent” conflicts with roommates or housemates. In another survey in the same country, 5.6\% of undergraduates reported that difficulties with their roommates have hampered their academic endeavors\cite{erb}. For comparison, 4\% of American students said that alcohol did the same thing\cite{erb}.

    We target college roommates because it has been proven that college roommate relationships have a momentous effect on a student's performance before and after graduation. Although the college-roommate phenomenon has existed for a long time, there is only one paper that congregates empirical information about college-roommate relationships\cite{erb}.

    \subsection{Proposed Solution}
    In order to maximize the number of compatible roommates on campus, we will be making a roommate-matching application for UPLB. Unlike other roommate matching applications that exist today, we will target students of \UPLB that are looking for roommates. Students of \UPLB  have different needs and preferences to working class people, so we cannot use existing roommate matching applications. In addition, there are no roommate matching applications that cater to the UPLB area or the Philippines. It is about time we have one for campus.
% Similar Applications
\section{Similar Applications}
\pubidadjcol
These applications have more similarities than differences. They ask the user if they are looking for a place to stay, or if they have a room and are looking for someone to stay with them. Then, the user is \textit{bombarded} with a long questionnaire that he has to fill out. It covers most of the basic criteria you would have when looking for a room. The union of questions asked in both websites can be seen in the appendices.

Both applications have similar questions, but have subtle insignificant differences. Roommates.com requires you to log in even before you start filling out the questionnaire. Roommie Match does not. However, in order to get a match, you will have to provide your credentials. Some people may find this surreptitious, as users will be forced togive their credentials after they have already filled out a long form.

Roommate.com\cite{roommates.com} is an application that caters to people looking for roommates in the United States of America. It can be accessed by going to \textit{www.roommates.com}. The home page lists cities where the appliication is used the most such as Las Vegas, Houston, Atlanta, Los Angeles, New York, etc. - all of which are American.

Anybody can create an account for free regardless of your location. Among its listed features for free accounts include: Photo profile, 2-way matching, and the freedom to contact potential roommates. The application has a premium account option, which is required to read messages sent to you by other users in the application. It costs \$5.99 for three days, roughly \$20 dollars for 30 days, and \$30 dollars for 60 days.
%  When you are registering, it asks if you are looking for a room or if you have a room. If you are looking for a room, it asks in which city and state you are looking for. Then it asks you how far from the city in miles you are willing to live in. Then it shows a list of cities which fit the criteria. From this list, you can select which cities it will search. Afterwards, you can select the regions of those cities it will look up (North, South, West, East, Central). Then it asks your preferred residence type (house, apartment, townhouse/condo, other), your move-in date, your maximum rent budget, duration, and whether or not you are willing to share a bedroom or a bathroom. After that, it asks about your preferred roommate characteristics - age, gender (all of it and all possible restrictions in the rainbow), smoking habits, cleanliness, pets (dogs, cats, caged animals, etc), children (if you are willing to live with children or not). Finally, it asks details about yourself -  age, sex, sexual orientation, occupation, smoking habits {indoor, outdoor}*, cleanliness, pets, children, and which emoticon in their list best represents you.
% \indent Once logged in as a user looking for a room, it shows a list of matching people that have rooms. You can see their age range, smoking habits, payment scheme (monthly, down payment, cost), location, availability date, utilities (electricity, gas, water), furnishing, nearby landmarks, nearby public transport hubs, existing occupants, etc. You can also see the profile of the person who owns the room. Premium members of the website can comment on profiles of other people. This can serve as a tool to thwart scammers and fraudulent users.

Another feature of Roommates.com is the powersearch feature. It is much like Google's advance search. The user will be redirected to a form that asks some questions (see appendix). Once the user has filled out the form, then a list of matching roommates will appear.

% For \"Who you are looking for\", Cleanliness, and Pets, the user can select more than one. The powersearch feature is meant to be used for a quick search since it asks for less information that when signing up.
%  \\
%  \\
% \indent There are pictures of the apartment, and the person looking for the apartment can post a picture of him/herself. A user can also opt to select a smiley face that best represents his personality. A user can only see the profiles of people he/she has matched with.
Roommate.com is not feasible for use in UPLB because there are no locations near the \UPLB that come up in the list of city selections. The form also forces the user to specify his state - a piece of information that is irrelevant for people living outside the United States of America. The website has a lot of javascript alerts and prompts when navigating, which is an anti-pattern to most modern web applications (as of this writing in 2016).
Roommiematch is another roommate match making algorithm that caters to some First-world countries.
Tinder is an application we have to review as well. In 2015, it has an estimated 50 million users\cite{tinderstat2}, 80\% of which were were millenials, or individuals born after 1982 \cite{tinderstat}\cite{millenial}.

\section{The Ideal Roommate Matching application}

% MATERIALS AND METHODS
\section{Materials and Methods}
% Irving's Algorithm
% Gale-Shapley's Algorithm
% \section{Proof of methods down here}

% RESULTS AND DISCUSSION
% \section{Results and Discussion}

% CONCLUSION AND FUTURE WORK
% \section{Conclusion and Future Work}

\newpage
% APPENDICES
\appendices
% Please look at the IEEE guidelines
\section{Roommate.com powersearch questions} %Mandatory title to specify an appendix
\begin{enumerate}
    \item Who you are looking for
    \begin{itemize}
        \item Straight female
        \item Professional
        \item Non Smoker
        \item Lesbian
        \item Student
        \item Smoker
        \item Straight male
        \item Unemployed
        \item Outside Smoker
        \item Gay male
        \item Military
        \item Retired
    \end{itemize}
    \item Age range
    \item Who
    \begin{itemize}
        \item Needs a place to live
        \item Has a place available
    \end{itemize}
    \item Location
    \begin{itemize}
        \item State
        \item City
    \end{itemize}
    \item Available (Date the room is available or the roomate is available to move)
    \item Payment (price range)
    \item Cleanliness
    \begin{itemize}
        \item Average
        \item Clean
        \item Messy
    \end{itemize}
    \item Pets
    \begin{itemize}
        \item No dogs
        \item No cats
        \item No caged pets
    \end{itemize}
    \item Whether or not you have a preference regarding children in the home
\end{enumerate}

% Room-8! Questions - To be updated more when we have more research on optimal roommate traits
\section{Questions and their Category in Room-8!}
\begin{enumerate}

    \item What are you looking for?
    \begin{itemize}
        \item A place with vacancies
        \item I have a room with vacancies
    \end{itemize}

    \item When
    \begin{itemize}
        \item Summers only?
        \item Start date
    \end{itemize}

    \item Utilities
    \begin{itemize}
        \item Air conditioning?
        \item Can I do my own laundry?
        \begin{itemize}
            \item Is there a washing machine?
            \item Is there a dryer?
            \item Is there a place to dry my clothes?
        \end{itemize}
        \item Can I cook? (Check all that apply)
        \begin{itemize}
            \item Gas Stove
            \item Electric Stove
            \item Microwave oven
            \item Water kettle
        \end{itemize}
        \item Internet connection
        \begin{itemize}
            \item What is the minimum speed requirement?
            \item Do you torrent?
        \end{itemize}
    \end{itemize}

    \item Hobbies
    \begin{itemize}
        \item Are you a gamer?
        \item Are you an activist?
        \item Are you into sports
    \end{itemize}

    \item Lifestyle
    \begin{itemize}
        \item Are you religious? / Are you a part of a cult?
        \item Do you urinate on the shower?
        \item What is an acceptable noise level for you? (1 - 10)
        \item Do you drink Alcohol?
        \item Do you smoke?
        \item Are you prefer to study in the early mornings or evenings?
        \item Do you have guests?
        \item Are you willing to share a bathroom?
        \item Do you have guests come over?
        \item Do you live with pets?
        \begin{itemize}
            \item Small caged animals (birds, hamsters)
            \item Not small caged animals (dogs)
        \end{itemize}
        \item Are you willing to share a bathroom?
    \end{itemize}

    \item Location Preferences
    \begin{itemize}
        \item Nearby Restaurants? (5 minute walk)
        \item How far from UPLB (How long a walk)
        \item General Area (List Barangays)
    \end{itemize}

    \item Misc
    \begin{itemize}
        \item Study area for guests?
        \item Easily Accessible
        \item Cost
        \item Payment scheme
        \item Curfew
    \end{itemize}
\end{enumerate}


\newpage
% ACKNOWLEDGMENT
\section*{Acknowledgment}
I would like to thank my adviser \ADVISER for her patience.

% BIOGRAPHY
\begin{biography}[{\includegraphics{./yourPicture.eps}}]{\ADVISEE}
\end{biography}


\newpage
% BIBLIOGRAPHY
% \bibliography{./cs190-ieee}
\bibliographystyle{./IEEE/IEEEtran}
\bibliography{marcoscmsc190sp}
% \nocite{*}
\end{document}
